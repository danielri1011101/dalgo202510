%%%%%%%%%%%%%
\documentclass{amsart}

\input{preamble}

\title{Definitions, Exercises, and More}

\author{Daniel R. Barrero R.}

\begin{document}

\maketitle

\begin{defn}
	Let $f : \NN \to \Rplus$.
	\begin{enumerate}
		\item We say $f$ is \emph{eventually non-decreasing} if there exists $n_0 \in \NN$ such that $f(n) \leq f(n+1)$ for all $n \geq n_0$.
		\item Let $b \geq 2$ be an integer. We say that $f$ is $b-$smooth if it is e.n.d. and $f(bn) \in O(f(n))$.
	\end{enumerate}
\end{defn}

\bigskip

% Is \n the first character of the current line, or the last character of the preceeding one? Maybe I can ask gpt...

\begin{defn}
	\begin{enumerate}
		\item _rooted tree_
		\item _root_
		\item _internal node_: a node that is not a leaf.
		\item _leaf_: a node with no children.
	\end{enumerate}
\end{defn}

\bigskip

\begin{defn}
	\begin{enumerate}
		\item The _height_ of a node is the length of the longest path from it to a leaf.
		\item The _depth_ of a node is the length of the unique path from the root to it.
		\item nodeLevel = treeHeight - nodeDepth.
	\end{enumerate}

\end{defn}

\bigskip

\begin{defn}
	A binary tree is _essentially complete_ if each of its internal nodes has _exactly two children_, with one possible exception: a node at depth (level?) 1, which lacks its left child. % draw many examples...
\end{defn}

\bigskip

\textbf{Claim.}

\begin{enumerate}
	\item If $f(n)$ is e.n.d., then $f(n) \in O(f(bn))$.
	% most likely part of exr 2.1.20...
\end{enumerate}

\end{document}
