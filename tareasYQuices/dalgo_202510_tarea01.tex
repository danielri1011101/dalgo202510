\documentclass{amsart}

%%%%%%%%%%%%%%%

\usepackage[utf8]{inputenc}
\usepackage[spanish]{babel}
\newcommand{\NN}{\mathbf{N}}
\newcommand{\ZZ}{\mathbf{Z}}
\newcommand{\QQ}{\mathbf{Q}}
\newcommand{\RR}{\mathbf{R}}
\newcommand{\Zpos}{\ZZ^{>0}}
\newcommand{\Rpos}{\RR^{>0}}
\newcommand{\la}{\langle}
\newcommand{\ra}{\rangle}


\title{isis1105 - 202510 \\ tarea 1}
\author{Daniel R. Barrero R.}
\date{\today}

\begin{document}

\maketitle

\textbf{0. Calentamiento.} Estos ejercicios no suman puntos de la tarea, son ejercicios fáciles para practicar el lenguaje formal.
\begin{enumerate}
	\item Demuestre que $O(f(n)) = O(g(n))$ si y sólo si $f \in O(g(n))$ y $g \in O(f(n))$.
\end{enumerate}

\bigskip

\textbf{I. Complejidad.}

\begin{enumerate}
	\item ¿Verdadero o Falso? justifique su respuesta con una demostración o un contraejemplo, según corresponda.
		\begin{enumerate}
			\item $2^{n+1} \in O(2^n)$
			\item Para cualquier función $f : \NN \to \Rplus$, si $f(n) \in O(n)$, entonces $f(n)^2 \in O(n^2)$.
			\item Para cualquier función $f : \NN \to \Rplus$, si $f(n) \in O(n)$, entonces $2^{f(n)} \in O(2^n)$.
		\end{enumerate}
	\item Demuestre que si $f \in O(g)$ y $g \in O(h)$, entonces $f \in O(h)$.
	\item De un ejemplo de funciones $f,g : \NN \to \Zpos$ tales que $f \notin O(g)$ y $g \notin O(f)$. Justifique su respuesta.
	\item Sean $f,g : \NN \to \Rpos$. Demuestre que
		\begin{enumerate}
			\item Si $\lim_{n \to \infty} f(n)/g(n) \in \Rpos$ entonces $O(f(n)) = O(g(n))$.
			\item Si $\lim_{n \to \infty} f(n)/g(n) = 0$ entonces $O(f(n)) \subsetneq O(g(n))$ y $O(g(n)) = O(g(n)+f(n))$.

		\end{enumerate}
	\item Demuestre que para cuales quiera $a,b>1$ $\log_a n \in O(\log_b n)$.
	\item ¿Es cierto que $2^{\log_a n} \in O(2^{\log_b n})$ para cualesquiera $a,b > 1$? Justifique su respuesta con una demostración o un contraejemplo.
\end{enumerate}

\bigskip

\textbf{2.} Determine el valor de verdad de las siguientes sentencias. Justifique su respuesta.

\begin{enumerate}
    \item $\exists n \ (2n = 3n)$
    \item $\exists n \ (n^2 = 2)$
\end{enumerate}

\vspace{1cm}

\textbf{3.} Determine el valor de verdad de cada una de estas sentencias. Justifique su respuesta.

\begin{enumerate}
    \item $\exists x (\forall y \ (xy = y))$
    \item $\exists x (\exists y \ ((x^2 > y) \land ( x < y )))$
    \item $\exists x (\forall y \ (x < y^2))$
    \item $\forall x (\exists y \ (x^2 < y))$
    \item $\exists x (\exists y \ (x^2 + y^2 = 5))$
    \item $\exists x (\exists y \ (x^2 + y^2 = 6))$
\end{enumerate}

\vspace{1cm}

\textbf{4.} Justifique en cada caso por qué la proposición es falsa.

\begin{enumerate}
    \item $\forall x (\forall y \ (x^2 = y^2 \to x = y))$
    \item $\forall x (\forall y \ (xy \geq x))$
    \item $\forall x (\exists y (y^2 - x < 100))$
    \item $\forall x (\forall y (x^2 \neq y^3))$
\end{enumerate}

\newpage

%%%%%%%%%%%%%

{\centering \textbf{DEFINICIONES}}

\begin{enumerate}
	\item $O(f(n)) = \{\tau : \NN \to \Rpos \ | \ 
		\exists c \in \RRplus, n_0 \in \NN : \ 
		\forall n \geq n_0 : \ \tau(n) \leq c f(n)\}$.
\end{enumerate}


\end{document}
